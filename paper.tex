% Options for packages loaded elsewhere
\PassOptionsToPackage{unicode}{hyperref}
\PassOptionsToPackage{hyphens}{url}
\documentclass[
]{article}
\usepackage{xcolor}
\usepackage[margin=1in]{geometry}
\usepackage{amsmath,amssymb}
\setcounter{secnumdepth}{-\maxdimen} % remove section numbering
\usepackage{iftex}
\ifPDFTeX
  \usepackage[T1]{fontenc}
  \usepackage[utf8]{inputenc}
  \usepackage{textcomp} % provide euro and other symbols
\else % if luatex or xetex
  \usepackage{unicode-math} % this also loads fontspec
  \defaultfontfeatures{Scale=MatchLowercase}
  \defaultfontfeatures[\rmfamily]{Ligatures=TeX,Scale=1}
\fi
\usepackage{lmodern}
\ifPDFTeX\else
  % xetex/luatex font selection
\fi
% Use upquote if available, for straight quotes in verbatim environments
\IfFileExists{upquote.sty}{\usepackage{upquote}}{}
\IfFileExists{microtype.sty}{% use microtype if available
  \usepackage[]{microtype}
  \UseMicrotypeSet[protrusion]{basicmath} % disable protrusion for tt fonts
}{}
\makeatletter
\@ifundefined{KOMAClassName}{% if non-KOMA class
  \IfFileExists{parskip.sty}{%
    \usepackage{parskip}
  }{% else
    \setlength{\parindent}{0pt}
    \setlength{\parskip}{6pt plus 2pt minus 1pt}}
}{% if KOMA class
  \KOMAoptions{parskip=half}}
\makeatother
\usepackage{graphicx}
\makeatletter
\newsavebox\pandoc@box
\newcommand*\pandocbounded[1]{% scales image to fit in text height/width
  \sbox\pandoc@box{#1}%
  \Gscale@div\@tempa{\textheight}{\dimexpr\ht\pandoc@box+\dp\pandoc@box\relax}%
  \Gscale@div\@tempb{\linewidth}{\wd\pandoc@box}%
  \ifdim\@tempb\p@<\@tempa\p@\let\@tempa\@tempb\fi% select the smaller of both
  \ifdim\@tempa\p@<\p@\scalebox{\@tempa}{\usebox\pandoc@box}%
  \else\usebox{\pandoc@box}%
  \fi%
}
% Set default figure placement to htbp
\def\fps@figure{htbp}
\makeatother
\setlength{\emergencystretch}{3em} % prevent overfull lines
\providecommand{\tightlist}{%
  \setlength{\itemsep}{0pt}\setlength{\parskip}{0pt}}
\usepackage{bookmark}
\IfFileExists{xurl.sty}{\usepackage{xurl}}{} % add URL line breaks if available
\urlstyle{same}
\hypersetup{
  pdftitle={STAT 400 Group 7 Paper DRAFT},
  pdfauthor={Group7},
  hidelinks,
  pdfcreator={LaTeX via pandoc}}

\title{STAT 400 Group 7 Paper DRAFT}
\author{Group7}
\date{}

\begin{document}
\maketitle

\section{Introduction}\label{introduction}

The population-attributable fraction (PAF) is used in epidemiology to
quantify how many cases of a disease can be attributed to a specific
exposure or the reduction in cases that could be achieves if the
exposure were to be eliminated
(\url{https://pmc.ncbi.nlm.nih.gov/articles/PMC1508384/?page=1}). PAF
can be defined as:

\[
PAF = \frac{P_e(RR-1)}{P_e(RR-1)+1}
\]

where \(P_e\) is the proportion of the population that is exposed and RR
is the relative risk, the ratio of the risk of disease among the exposed
group to the risk among the unexposed group
(\url{https://pmc.ncbi.nlm.nih.gov/articles/PMC1508384/?page=3}).

While point estimates of the population-attributable fraction (PAF) are
informative, they do not express uncertainty arising from estimation of
exposure prevalence and relative risk. Confidence intervals (CIs) are
therefore essential for interpreting PAF estimates and for informing
public health decision-making. However, because the PAF is a proportion,
it is logically bounded between 0 and 1. Standard CI methods based on
large-sample normal approximations may violate these bounds, producing
negative values or values exceeding one, which are not interpretable for
epidemiology. This limitation motivates the comparison of alternative CI
methods that better account for the nonlinearity and bounded nature of
the PAF.

The article \emph{A Comparison of Green, Delta, and Monte Carlo Methods
to Select an Optimal Approach for Calculating the 95\% Confidence
Interval of the Population-Attributable Fraction.} by Lee et
al.~compared three methods for computing confidence intervals for PAFs:

\begin{itemize}
\tightlist
\item
  Green method\\
\item
  Delta method\\
\item
  Monte Carlo method
\end{itemize}

This project aims to replicate two of the scenarios used in the article
and extend this by evaluating how often the three methods produce
intervals that are outside of the logical bound of {[}0,1{]}, which
would indicate a proportion of the cases either less than 0\% or greater
than 100\%.

\section{Methods}\label{methods}

\subsection{Simulation Overview}\label{simulation-overview}

We replicated the simulation design from the article \emph{A Comparison
of Green, Delta, and Monte Carlo Methods\ldots{}} to estimate the
Population-Attributable Fraction (PAF) and compare three confidence
interval (CI) methods: the Delta method, the Greenland (Green) method,
and the Monte Carlo method based on two scenarios in Lee et al.~(2024):

-Scenario A: Modest effect size (\(RR\) = 1.2), moderate prevalence
(\(P_e\) = 0.10), sample size \(T_p\) = 1,000.

-Scenario B: Strong effect size (\(RR\) = 5.0), same prevalence, large
sample size \(T_p\) = 100,000.

Then a simulation study with \(n = 500\) iterations was performed on
each scenario to determine the proportion of CIs containing the true
PAF, average CI width, and the the proportion of CIs that violated the
{[}0,1{]} boundary.

\subsection{Notations and definitions in this
study}\label{notations-and-definitions-in-this-study}

\begin{itemize}
\tightlist
\item
  \textbf{PAF}: Population-attributable fraction\\
\item
  \(P_e\): Prevalence of exposure in the total population\\
\item
  \textbf{RR}: Relative risk for the exposed group\\
\item
  \(T_p\): Total population size\\
\item
  \textbf{V{[}\textbf{\(\beta\)}{]}}: Variance of the log(RR)
  estimator\\
\item
  \textbf{V{[}\textbf{\(P_e\)}{]}}: Variance of the exposure prevalence
  estimator\\
\item
  \textbf{CI}: 95\% confidence interval for the PAF
\end{itemize}

\subsection{PAF Formula}\label{paf-formula}

The PAF was estimated using the Levin formula, which depends on the
exposure prevalence and the risk ratio:

\[
PAF = \frac{Pe(RR - 1)}{Pe(RR - 1) + 1}
\]

This formula is implemented in the function \texttt{paf\_single()}.

\begin{center}\rule{0.5\linewidth}{0.5pt}\end{center}

\subsection{Confidence Interval
Methods}\label{confidence-interval-methods}

\subsubsection{Delta Method}\label{delta-method}

The Delta method uses a Taylor series approximation to estimate the
variance of the PAF estimator.\\
The variance of PAF is computed using the variances of Pe and β
(log(RR)), following the structure:

\[
V[\widehat{PAF}] = \frac{V[Pe](RR - 1)^2 + V[β](RR \cdot Pe)^2}{(Pe(RR - 1) + 1)^4}
\]

A 95\% CI is then constructed as:

\[
PAF \pm 1.96 \times \sqrt{V[\widehat{PAF}]}
\]

This procedure is implemented in \texttt{delta\_paf\_ci()}.

\subsubsection{Greenland (Green) Method}\label{greenland-green-method}

The Greenland method uses a variance-stabilizing transformation of
\(1 - PAF\), incorporating the odds of exposure:

\[
O = \frac{Pe}{1 - Pe}
\]

This method accounts for variability in both the RR and Pe.\\
After computing the transformed variance, the 95\% CI is
back-transformed to the PAF scale:

\[
CI = 1 - (1 - PAF) \exp(\pm 1.96 \sqrt{V[\widehat{PAF}]})
\]

Implemented in \texttt{greenland\_paf\_ci()}.

\subsubsection{Monte Carlo Method}\label{monte-carlo-method}

The Monte Carlo method generates random samples for log(RR) and Pe:

\begin{itemize}
\tightlist
\item
  \(\beta = \log(RR)\) is sampled from a normal distribution with
  variance V{[}β{]}\\
\item
  Pe is sampled from a binomial distribution with variance V{[}Pe{]}
\end{itemize}

For each simulation, a new PAF is computed.\\
The 95\% CI is defined as the 2.5th and 97.5th percentiles of the
simulated PAF distribution.

Implemented in \texttt{mc\_paf\_ci()}.

\subsection{Simulation Design}\label{simulation-design}

For each scenario, we:

\begin{enumerate}
\def\labelenumi{\arabic{enumi}.}
\tightlist
\item
  Drew RR* from a log-normal distribution and Pe* from a binomial
  distribution.
\item
  Computed CIs using the three methods.
\item
  Recorded Ci coverage and width.
\item
  Calculated the proportion of CIs that had values outside of our
  logical boundaries.
\end{enumerate}

\section{Results}\label{results}

Scenario A had a modest effect size (RR = 1.2), moderate prevalence (Pe
= 0.10), sample size T\_p = 1,000, and a true PAF = 0.0196

\begin{verbatim}
##        Method        PAF       Lower      Upper
## 1       Delta 0.01960784 0.004060779 0.03515491
## 2   Greenland 0.01960784 0.003869669 0.03509736
## 3 Monte Carlo 0.01955304 0.005364261 0.03654544
\end{verbatim}

\begin{verbatim}
##        Method Coverage      Width
## 1       Delta    0.956 0.03131707
## 2   Greenland    0.956 0.03145046
## 3 Monte Carlo    0.952 0.03148512
\end{verbatim}

\begin{verbatim}
##        Method Prob_Outside_Logical_Range
## 1       Delta                      0.262
## 2   Greenland                      0.266
## 3 Monte Carlo                      0.216
\end{verbatim}

\begin{verbatim}
##        Method       PAF     Lower     Upper
## 1       Delta 0.2857143 0.2520350 0.3193936
## 2   Greenland 0.2857143 0.2512268 0.3186133
## 3 Monte Carlo 0.2857729 0.2526543 0.3199357
\end{verbatim}

\begin{verbatim}
##        Method Coverage      Width
## 1       Delta    0.954 0.03128194
## 2   Greenland    0.952 0.03141555
## 3 Monte Carlo    0.944 0.03138405
\end{verbatim}

\begin{verbatim}
##        Method Prob_Outside_Logical_Range
## 1       Delta                      0.256
## 2   Greenland                      0.270
## 3 Monte Carlo                      0.208
\end{verbatim}

\section{Discussion}\label{discussion}

This study compared three methods for constructing confidence intervals
for the population-attributable fraction (PAF): the Delta method, the
Greenland method, and a Monte Carlo simulation--based approach. Using
two parameter scenarios taken directly from Lee et al.~(2024), we
evaluated how these methods behave under small PAF values with modest
effect size (Scenario A) and large PAF values with strong effect size
(Scenario B). Across both settings, our results were consistent with the
patterns reported by Lee et al., supporting their conclusions.

Our extension analysis further emphasized the importance of
boundary-aware CI methods. By evaluating how often each method generated
values outside the {[}0,1{]} range, we showed that Monte Carlo provides
the best protection against invalid intervals in both mild (Scenario A)
and extreme (scenario B) parameter settings. Our simulation suggests
that the Monte Carlo method provides the most reliable confidence
interval estimation for the population-attributable fraction.

\section{References}\label{references}

ADD FOR FINAL

\end{document}
